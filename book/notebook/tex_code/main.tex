%!TEX TS-program = xelatex
%!TEX encoding = UTF-8 Unicode

\documentclass[8pt]{article}
\usepackage[a4paper]{geometry}
\usepackage[english]{babel}
\usepackage{amssymb,amsthm,amsmath}
%\usepackage{xltxtra}
%\usepackage{stmaryrd}
\usepackage{graphicx}
\usepackage{listings}
\usepackage{color}
\lstset{
	extendedchars=true,
	showstringspaces=false,
	escapeinside=``,
	keywordstyle=\color{blue},
	commentstyle=\color[rgb]{0.133,0.545,0.133},
	columns=flexible,
	language=python,
	tabsize=2,
	basicstyle=\normalsize\selectfont\ttfamily,
	numbers=left,
	frame=lines,
	breaklines=true
}
\geometry{
	left=15mm,
	right=7mm,
	top=7mm,
	bottom=15mm
}
\usepackage{multicol}
\setlength{\columnsep}{1cm}

\title{SWERC NoteBook}
\author{SaintGermainDesPrés : Mathilde Bonin, Eyal Cohen, Hugo Demaret}
\date\today

\begin{document}
    \maketitle
    \section{Configuration}
        \subsection{C/C++}
    \section{Chaînes de caractères}
    \section{Séquences}
    \section{Parcours de graphes}
        \subsection{DFS - Depth First Search}
        {\scriptsize\lstinputlisting{../code/graphes/dfs.py}}
        \subsection{BFS - Breadth First Search}
        {\scriptsize\lstinputlisting{../code/graphes/bfs.py}}
        \subsection{Topological Sort}
        \subsection{Composantes connexes}
        \subsection{Composantes bi-connexe}
        \subsection{Composantes fortement connexe}
        \subsection{2-SAT}
        \subsection{Postier Chinois}
        \subsection{Chemin eulérien}
        \subsection{Chemin le plus court}
            \subsubsection{Poids positif ou nul - Dijkstra}
            \subsubsection{Poids arbitraire - Bellman-Ford}
            \subsubsection{Floyd-Warshall}
    \section{Points et polygones}
        \subsection{Points}
            \subsubsection{Points}
            {\scriptsize\lstinputlisting{../code/points/point.py}}
            \subsubsection{Cross-product}
            {\scriptsize\lstinputlisting{../code/points/crossproduct.py}}
            \subsubsection{Direction}
            {\scriptsize\lstinputlisting{../code/points/direction.py}}
        \subsection{Enveloppe convexe}
            Complexité : $\mathcal{O}(n \log(n))$
            {\scriptsize\lstinputlisting{../code/points/jarvismarch.py}}
        \subsection{Aire d'un polygone}
        \textit{Uniquement pour les polygones simples. Réduire à des composantes simples sinon.}
        $A = \frac{1}{2} \sum_{i=0}^{n-1}\left(x_{i}x_{i+1} - x_{i+1}y_{i}\right)$
        {\scriptsize\lstinputlisting{../code/points/polygonearea.py}}
        \subsection{Points entiers dans un polygone}
            \subsubsection{Sur le contour}
            \subsubsection{Dans le polygone}
                Théorème de Pick :
                $P = n_{i} + \frac{n_{b}}{2}-1$
        \subsection{Paire de points les plus proches}
    \section{Ensembles}
        \subsection{Rendu de monnaie}
        Problème NP-Complet.
        \subsection{Sac à dos}
        Problème NP-Complet.
        {\scriptsize\lstinputlisting{../code/ensembles/knapsack.py}}
        \subsection{k-somme}
    \section{Calculs}
        \subsection{PGCD}
        {\scriptsize\lstinputlisting{../code/calculs/pgcd.py}}
        \subsection{Coefficients de Bézout}
        {\scriptsize\lstinputlisting{../code/calculs/bezout.py}}
        \subsection{Coefficients binomiaux}
        {\scriptsize\lstinputlisting{../code/calculs/binom.py}}
\end{document}